\documentclass{article}

\usepackage{float}
\usepackage[a4paper, margin=2cm]{geometry}

\begin{document}

% Title Page
\begin{titlepage}
    \hrule height 0.1cm
    \vspace{40pt}
    \begin{flushright}
        \Huge\textbf{Software Requirement\\Specification\\}
        \vspace{30pt}
        \Large\textbf{for\\}
        \vspace{25pt}
        \huge\textbf{Test Case\\Generator\\}
        \vspace{25pt}
        \Large\textbf{version: 0.0}
        \vspace{50pt}

        \vfill
        \begin{Large}
            \textbf{Prepared by:\\}
            Anjali Bansal\\
            Ankit Gourav\\
            Dipehs Mangrora\\
            Sourav Jaiswal\\
        \end{Large}
        \vfill
    \end{flushright}

    % Print date on which doc is generated.
    \texttt{\today}
\end{titlepage}
\pagebreak

% Page 1
\tableofcontents
\vfill

\begin{table}[H]
    \begin{flushleft}
        \Large\textbf{Revisions}
    \end{flushleft}
    \vspace{10pt}\normalsize
    \def\arraystretch{1.2}
    \centering
    \setlength{\tabcolsep}{25pt}
    \begin{tabular}{|c|c|c|c|}
        \hline
        Version & Primary Author(s) & Description of version & Date Completed\\
        \hline\hline
        0.0 & & & \\
        \hline
        & & & \\
        \hline
    \end{tabular}
\end{table}
\pagebreak

\section{Introduction}
\subsection{Document Purpose}
\paragraph{}
This document is prepared in order to determine the software requirements for the
\textbf{Test Case Generator} platform. \textbf{Test Case Generator} platform is a
web-based platform for generating test cases where developers can generate test for
different alogorithms and functions. The platform allows developers to generate
scalable test cases suitable for their work. In order to gain an overview about
the report, firstly, the purpose and scope of this document will be given, and the
overall description of \textbf{Test Case Generator} platform system is followed. In
addition to these, system feature such as custom test case generation, large test case
generation, login service, etc. are described deeply. After mentioning about the
introduction of the platform, the specific requirements will be addressed for it.
In the final part, functional and non-functional requirements will be addressed.

\subsection{Document Convention}
\paragraph{}
This Document was created based on the IEEE template for System Requirement Specification
Documents.

\subsection{Intended Users}
\paragraph{}
This platform is Intended towards the students who are practicing programming on online
platforms where they cannot test problems with large test cases before code submission,
and towards developers who are unable to test the full extent of their applications.

\section{Overall Description}
\subsection{Product Perspective} 
\paragraph{}
Test Generator is developed to allow coders to test their codes for different scenarios. It reduces coder's efforts and saves their time.

\subsection{Production Functions} 
\begin{itemize}
    \item The system should allow a user to select a option among various data types.
    \item The system should allow a user to give a range for result.
    \item The system should allow a user to choose result in different forms.
\end{itemize}

\end{document}
