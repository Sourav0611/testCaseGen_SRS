\documentclass{article}

\usepackage{float}
\usepackage[a4paper, margin=2cm]{geometry}

\begin{document}

% Title Page
\begin{titlepage}
    \hrule height 0.1cm
    \vspace{40pt}
    \begin{flushright}
        \Huge\textbf{Software Requirement\\Specification\\}
        \vspace{30pt}
        \Large\textbf{for\\}
        \vspace{25pt}
        \huge\textbf{Test Case\\Generator\\}
        \vspace{25pt}
        \Large\textbf{version: 0.0}
        \vspace{50pt}

        \vfill
        \begin{Large}
            \textbf{Prepared by:\\}
            Anjali Bansal\\
            Ankit Gourav\\
            Dipesh Mangrora\\
            Sourav Jaiswal\\
        \end{Large}
        \vfill
    \end{flushright}

    % Print date on which doc is generated.
    \texttt{\today}
\end{titlepage}
\pagebreak

% Page 1
\tableofcontents
\vfill

\begin{table}[H]
    \begin{flushleft}
        \Large\textbf{Revisions}
    \end{flushleft}
    \vspace{10pt}\normalsize
    \def\arraystretch{1.2}
    \centering
    \setlength{\tabcolsep}{25pt}
    \begin{tabular}{|c|c|c|c|}
        \hline
        Version & Primary Author(s) & Description of version & Date Completed\\
        \hline\hline
        0.0 & & & \\
        \hline
        & & & \\
        \hline
    \end{tabular}
\end{table}
\pagebreak

\section{Introduction}
\subsection{Document Purpose}
\paragraph{}
This document is prepared in order to determine the software requirements for the
\textbf{Test Case Generator} platform. \textbf{Test Case Generator} platform is a
web-based platform for generating test cases where developers can generate test
for different algorithms and functions. The platform allows developers to generate
scalable test cases suitable for their work. This document outlines the scope,
objectives, functional and non-functional requirements of the platform, serving as
a reference for project developers and testers.

\subsection{Scope}
\paragraph{}
The scope of TestGen Pro includes the development of a web-based application and a
command-line interface (CLI) for test case generation, review, and management. The
software will be designed to support test automation and integration into CI/CD
pipelines.

\subsection{Intended Users}
\begin{itemize}
    \item Testers
    \item Developers
    \item Programmers
    \item Project Managers
    \item Technical Writers
    \item Educators and Students
    \item Oepn Source Contributors
    \item Test Automation Engineers
\end{itemize}

\subsection{Definitions, Acronyms, and Abbreviations}
\begin{itemize}
    \item SRS: Software Requirements Specification
    \item CI/CD: Continuous Integration/Continuous Deployment
    \item API: Application Programming Interface
    \item UI: User Interface
\end{itemize}

\subsection{Document Convention}
\paragraph{}
This Document was created based on the IEEE template for System Requirement
Specification Documents.

\section{Overall Description}
\subsection{Product Perspective}
\paragraph{}
Test Generator is developed to allow coders to test their codes for different scenarios.
It reduces coder's efforts and saves their time.

\subsection{Production Functions}
\begin{itemize}
    \item The system should allow a user to select a option among various data types.
    \item The system should allow a user to give a range for result.
    \item The system should allow a user to choose result in different forms.
\end{itemize}

\section{External Interface Requirements}
\subsection{User Interfaces}
\paragraph{}
Go to homepage of our website in a browser on your system. From menu list available
above choose which type of test cases you want to generate.This opens specific test
case generator for chosen type.Now enter further details and press enter.This will
generate test cases for you.

\subsection{Hardware Interfaces}
\begin{itemize}
    \item Mobile devices
    \item Laptop, PCs
\end{itemize}

\subsection{Software Interfaces}
\begin{itemize}
    \item Android 5.0 and above
    \item Linux (Ubuntu, Fedora, etc.)
    \item macOS
    \item Microsoft Windows (XP and above)
\end{itemize}

\subsection{Communication Interfaces}
\paragraph{}
Requires an internet connection (Ethernet/WIFI).

\section{Functional Requirements:}

\subsection{User Authentication and Authorization:}
\paragraph{}
Users must be able to create accounts and log in.
User roles (e.g., admin, tester, developer) should determine access to various features.
Administrators can manage user roles and permissions.

\subsection{Test Case Generation:}
\paragraph{}
Users should be able to define test case generation parameters, such as input data,
constraints, and coverage criteria. The software must generate test cases based on the
specified criteria. Users can trigger test case generation manually or automatically
through integration with CI/CD pipelines.

\subsection{Test Case Review and Approval:}
\paragraph{}
If applicable, users can review and approve generated test cases before execution.\\
Test cases can be marked as approved or rejected.

\subsection{Test Case Export:}
\paragraph{}
Users can export generated test cases in common formats (e.g., CSV, JSON, XML).\\
Exported test cases must include detailed information about the test case scenarios.

\subsection{Data Import and Integration:}
\paragraph{}
Support for importing existing test data and test case templates from various sources (e.g., files, databases).\\
Integration with other test management or test automation tools through APIs.

\subsection{Command-Line Interface (CLI):}
\paragraph{}
Provide a command-line interface for users who prefer working in a terminal environment.\\
CLI must offer the same functionality as the graphical user interface.

\subsection{Error Handling:}
\paragraph{}
The system should gracefully handle errors and provide informative error messages to users.
Errors should be logged for debugging purposes.

\section{Product Convention}
\subsection{Project Naming Convention:}
\paragraph{}
Choose a clear and concise name for the project, avoiding special characters or spaces.
Use camelCase or snake\_case for multi-word project names.
Example\: testCaseGenerator or test\_case\_generator

\subsection{Versioning:}
\paragraph{}
Follow Semantic Versioning (SemVer) for version numbers (e.g., MAJOR.MINOR.PATCH).
Start with version 1.0.0 for the initial release.

\subsection{Coding Standards:}
Follow a consistent coding style guide (e.g., PEP 8 for Python, or the appropriate
language-specific guide). Use meaningful variable and function names for improved
code readability.

\subsection{Documentation:}
\paragraph{}
Provide detailed documentation for installation, usage, and maintenance.Use a
documentation generator tool (e.g., Sphinx or Doxygen) for creating user and developer
documentation.

\subsection{Testing and Test Cases:}
\begin{itemize}
    \item Implement unit tests to ensure code reliability.
    \item Maintain a separate test suite or directory.
    \item Create and document sample test cases for end-users.
\end{itemize}

\subsection{Source Code Management:}
\begin{itemize}
    \item Use a version control system (e.g., Git).
    \item Host the project repository on a platform like GitHub or GitLab.
    \item Adhere to a branching strategy (e.g., feature branching).
\end{itemize}

\subsection{Licensing:}
\paragraph{}
Choose an appropriate open-source license (e.g., MIT, Apache License, or GPL) and
include it in your project's repository.

\subsection{Dependencies and Package Management:}
Use a package manager (e.g., pip for Python) to manage project dependencies.
Include a requirements.txt file to specify required packages and their versions.
Error Handling and Logging:
\begin{itemize}
    \item Implement consistent error handling strategies.
    \item Use logging libraries for debugging and error reporting.
\end{itemize}

\subsection{Security:}
\begin{itemize}
    \item Regularly update dependencies to patch known vulnerabilities.
    \item Perform security audits and code reviews periodically.
\end{itemize}

\subsection{User Interface (UI) Guidelines:}
\begin{itemize}
    \item If the software includes a UI, adhere to design and usability guidelines.
    \item Ensure cross-browser and cross-platform compatibility.
\end{itemize}

\subsection{Performance Optimization:}
\begin{itemize}
    \item Continuously monitor and optimize software performance.
    \item Profile code to identify and address bottlenecks.
\end{itemize}

\subsection{Release and Deployment:}
\begin{itemize}
    \item Follow a deployment strategy (e.g., continuous integration/continuous
        deployment) to streamline releases.
    \item Document the release process and maintain release notes.
\end{itemize}

\subsection{User Support and Feedback:}
\begin{itemize}
    \item Establish channels for user feedback and support.
    \item Respond promptly to user inquiries and issues.
\end{itemize}

\subsection{Quality Assurance:}
\begin{itemize}
    \item Implement code review processes to maintain code quality.
    \item Use automated code analysis tools (e.g., linters, static analyzers) where applicable.
\end{itemize}

\subsection{Code Repository Structure:}
\paragraph{}
Organize code, documentation, and related files into a structured directory layout for easy navigation.

\subsection{Contributor Guidelines:}
\paragraph{}
If the project is open source, provide contribution guidelines for external contributors.

\subsection{Continuous Improvement:}
\paragraph{}
Encourage continuous improvement through retrospectives and feedback loops.

\end{document}
